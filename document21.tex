\documentclass[journal,12pt,onecolumn]{IEEEtran}
%
\usepackage{setspace}
\usepackage{gensymb}
%\doublespacing
\singlespacing

%\usepackage{graphicx}
%\usepackage{amssymb}
%\usepackage{relsize}
\usepackage[cmex10]{amsmath}
%\usepackage{amsthm}
%\interdisplaylinepenalty=2500
%\savesymbol{iint}
%\usepackage{txfonts}
%\restoresymbol{TXF}{iint}
%\usepackage{wasysym}
\usepackage{amsthm}
%\usepackage{iithtlc}
\usepackage{mathrsfs}
\usepackage{txfonts}
\usepackage{stfloats}
\usepackage{bm}
\usepackage{cite}
\usepackage{cases}
\usepackage{subfig}
%\usepackage{xtab}
\usepackage{longtable}
\usepackage{multirow}
%\usepackage{algorithm}
%\usepackage{algpseudocode}
\usepackage{enumitem}
\usepackage{mathtools}
\usepackage{steinmetz}
\usepackage{tikz}
\usepackage{circuitikz}
\usepackage{verbatim}
\usepackage{tfrupee}
\usepackage[breaklinks=true]{hyperref}
%\usepackage{stmaryrd}
\usepackage{tkz-euclide} % loads  TikZ and tkz-base
%\usetkzobj{all}
\usetikzlibrary{calc,math}
\usepackage{listings}
    \usepackage{color}                                            %%
    \usepackage{array}                                            %%
    \usepackage{longtable}                                        %%
    \usepackage{calc}                                             %%
    \usepackage{multirow}                                         %%
    \usepackage{hhline}                                           %%
    \usepackage{ifthen}                                           %%
  %optionally (for landscape tables embedded in another document): %%
    \usepackage{lscape}     
\usepackage{multicol}
\usepackage{chngcntr}
%\usepackage{enumerate}

%\usepackage{wasysym}
%\newcounter{MYtempeqncnt}
\DeclareMathOperator*{\Res}{Res}
%\renewcommand{\baselinestretch}{2}
\renewcommand\thesection{\arabic{section}}
\renewcommand\thesubsection{\thesection.\arabic{subsection}}
\renewcommand\thesubsubsection{\thesubsection.\arabic{subsubsection}}

\renewcommand\thesectiondis{\arabic{section}}
\renewcommand\thesubsectiondis{\thesectiondis.\arabic{subsection}}
\renewcommand\thesubsubsectiondis{\thesubsectiondis.\arabic{subsubsection}}

% correct bad hyphenation here
\hyphenation{op-tical net-works semi-conduc-tor}
\def\inputGnumericTable{}                                 %%

\lstset{
%language=C,
frame=single, 
breaklines=true,
columns=fullflexible
}
%\lstset{
%language=tex,
%frame=single, 
%breaklines=true
%}

\begin{document}
%


\newtheorem{theorem}{Theorem}[section]
\newtheorem{problem}{Problem}
\newtheorem{proposition}{Proposition}[section]
\newtheorem{lemma}{Lemma}[section]
\newtheorem{corollary}[theorem]{Corollary}
\newtheorem{example}{Example}[section]
\newtheorem{definition}[problem]{Definition}
%\newtheorem{thm}{Theorem}[section] 
%\newtheorem{defn}[thm]{Definition}
%\newtheorem{algorithm}{Algorithm}[section]
%\newtheorem{cor}{Corollary}
\newcommand{\BEQA}{\begin{eqnarray}}
\newcommand{\EEQA}{\end{eqnarray}}
\newcommand{\define}{\stackrel{\triangle}{=}}

\bibliographystyle{IEEEtran}
%\bibliographystyle{ieeetr}


\providecommand{\mbf}{\mathbf}
\providecommand{\pr}[1]{\ensuremath{\Pr\left(#1\right)}}
\providecommand{\qfunc}[1]{\ensuremath{Q\left(#1\right)}}
\providecommand{\sbrak}[1]{\ensuremath{{}\left[#1\right]}}
\providecommand{\lsbrak}[1]{\ensuremath{{}\left[#1\right.}}
\providecommand{\rsbrak}[1]{\ensuremath{{}\left.#1\right]}}
\providecommand{\brak}[1]{\ensuremath{\left(#1\right)}}
\providecommand{\lbrak}[1]{\ensuremath{\left(#1\right.}}
\providecommand{\rbrak}[1]{\ensuremath{\left.#1\right)}}
\providecommand{\cbrak}[1]{\ensuremath{\left\{#1\right\}}}
\providecommand{\lcbrak}[1]{\ensuremath{\left\{#1\right.}}
\providecommand{\rcbrak}[1]{\ensuremath{\left.#1\right\}}}
\theoremstyle{remark}
\newtheorem{rem}{Remark}
\newcommand{\sgn}{\mathop{\mathrm{sgn}}}
\providecommand{\abs}[1]{\left\vert#1\right\vert}
\providecommand{\res}[1]{\Res\displaylimits_{#1}} 
\providecommand{\norm}[1]{\left\lVert#1\right\rVert}
%\providecommand{\norm}[1]{\lVert#1\rVert}
\providecommand{\mtx}[1]{\mathbf{#1}}
\providecommand{\mean}[1]{E\left[ #1 \right]}
\providecommand{\fourier}{\overset{\mathcal{F}}{ \rightleftharpoons}}
%\providecommand{\hilbert}{\overset{\mathcal{H}}{ \rightleftharpoons}}
\providecommand{\system}{\overset{\mathcal{H}}{ \longleftrightarrow}}
	%\newcommand{\solution}[2]{\textbf{Solution:}{#1}}
\newcommand{\solution}{\noindent \textbf{Solution: }}
\newcommand{\cosec}{\,\text{cosec}\,}
\providecommand{\dec}[2]{\ensuremath{\overset{#1}{\underset{#2}{\gtrless}}}}
\newcommand{\myvec}[1]{\ensuremath{\begin{pmatrix}#1\end{pmatrix}}}
\newcommand{\mydet}[1]{\ensuremath{\begin{vmatrix}#1\end{vmatrix}}}
%\numberwithin{equation}{section}
\numberwithin{equation}{subsection}
%\numberwithin{problem}{section}
%\numberwithin{definition}{section}
\makeatletter
\@addtoreset{figure}{problem}
\makeatother

\let\StandardTheFigure\thefigure
\let\vec\mathbf
%\renewcommand{\thefigure}{\theproblem.\arabic{figure}}
\renewcommand{\thefigure}{\theproblem}
%\setlist[enumerate,1]{before=\renewcommand\theequation{\theenumi.\arabic{equation}}
%\counterwithin{equation}{enumi}


%\renewcommand{\theequation}{\arabic{subsection}.\arabic{equation}}

\def\putbox#1#2#3{\makebox[0in][l]{\makebox[#1][l]{}\raisebox{\baselineskip}[0in][0in]{\raisebox{#2}[0in][0in]{#3}}}}
     \def\rightbox#1{\makebox[0in][r]{#1}}
     \def\centbox#1{\makebox[0in]{#1}}
     \def\topbox#1{\raisebox{-\baselineskip}[0in][0in]{#1}}
     \def\midbox#1{\raisebox{-0.5\baselineskip}[0in][0in]{#1}}

\vspace{3cm}


\title{Assignment 17}
\author{Jayati Dutta}





% make the title area
\maketitle

%\newpage

%\tableofcontents

\bigskip

\renewcommand{\thefigure}{\theenumi}
\renewcommand{\thetable}{\theenumi}
%\renewcommand{\theequation}{\theenumi}


\begin{abstract}
This is a simple document explaining how to determine the upper and lower limits of the dimension of a vector space which is the intersection of 3 subspaces of a vector space and also to check whether the former vector space is a subspace of that vector space or not.
\end{abstract}

%Download all python codes 
%
%\begin{lstlisting}
%svn co https://github.com/JayatiD93/trunk/My_solution_design/codes
%\end{lstlisting}

Download all and latex-tikz codes from 
%
\begin{lstlisting}
svn co https://github.com/gadepall/school/trunk/ncert/geometry/figs
\end{lstlisting}
%


\section{Problem}
Let $\vec{W_1}$, $\vec{W_2}$, $\vec{W_3}$ be 3 distinct subspaces of $\vec{R}^{10}$ such that each $\vec{W_i}$ has dimension of 9. Let $\vec{W} = \vec{W_1} \cap \vec{W_2} \cap \vec{W_3}$. Then we can conclude that\\
1. $\vec{W}$ may not be a subspace of $\vec{R}^{10}$\\
2. dim $\vec{W} \leq 8$\\
3. dim $\vec{W} \geq 7$\\
4. dim $\vec{W} \leq 3$\\
 

\section{Solution} 
\begin{longtable}{|c|c|}
\hline
\multirow{3}{*}{} & \\
$\textbf{Given}$ & $\vec{W_1}$, $\vec{W_2}$, $\vec{W_3}$\\
& are 3 distinct subspaces of \\
& $\vec{R}^{10}$\\
& \\
& Each $\vec{W_i}$ has dimension 9\\
& \\
& $\vec{W} = \vec{W_1} \cap \vec{W_2} \cap \vec{W_3}$\\
& \\
\hline
\multirow{3}{*}{} & \\
\textbf{Statement1} & $\vec{W}$ may not be a subspace of\\
& $\vec{R}^{10}$\\
\hline
Explanation & As $\vec{W} = \vec{W_1} \cap \vec{W_2}\cap \vec{W_3}$\\
& and $\vec{W_1}$, $\vec{W_2}$, $\vec{W_3}$ \\
& are subspaces of $\vec{W}$,then $\vec{W}$\\
& must be a subspace of $\vec{R}^{10}$.\\
& So the first option is false.\\
\hline
\multirow{3}{*}{} & \\
\textbf{Statement2} & dim $\vec{W} \leq 8$\\
\hline
Explanation & As $\vec{W}$ be a subspace of a \\
& finite dimension vector space $\vec{R}^{10}$ \\
& and dim $\vec{R}^{10}$= 10, so $\vec{W}$ \\
& is finite dimension and \\
& dim $\vec{W} \leq 10$ \\
& \\
\hline
$\textbf{Theorem}$ & dim ($\vec{W_1} \cap \vec{W_2}$)\\
& = dim($\vec{W_1}$)+dim($\vec{W_2}$)-dim($\vec{W_1}+\vec{W_2}$)\\
& and \\
& $\vec{W_1} \cap \vec{W_2}$ is also a subspace of $\vec{R}^{10}$\\
& \\
\hline
\multirow{3}{*}{} & \\
\textbf{Proof} & The minimum dimension of \\
& $\vec{W} = \vec{W_1} \cap \vec{W_2} \cap \vec{W_3}$ \\
\hline
Explanation & Let us consider $\vec{V}=\vec{R}^{10}$ and $dim(\vec{V})=10$\\
& and $\vec{U} = \vec{W_1} \cap \vec{W_2}$\\
& So, $dim(\vec{W_1} \cap \vec{W_2} \cap \vec{W_3}$) = $dim(\vec{U}$)\\
& +$dim(\vec{W_3}$) - $dim(\vec{U}+\vec{W_3}$)\\
& \\
& $\textbf{or,}$ $dim(\vec{W_1} \cap \vec{W_2} \cap \vec{W_3}$) = $dim(\vec{W_1}$)\\
& +$dim(\vec{W_2}$)+$dim(\vec{W_3}$) - $dim(\vec{W_1}+\vec{W_1}$)\\
& -$dim((\vec{W_1} \cap \vec{W_2})+\vec{W_3}$)\\
& \\
\hline
& \\
& Now, $(\vec{W_1} \cap \vec{W_2})+\vec{W_3} \subseteq \vec{V}$\\
& $\implies$ $dim((\vec{W_1} \cap \vec{W_2})+\vec{W_3})\leq dim(\vec{V})$\\
& $\implies$ -$dim((\vec{W_1} \cap \vec{W_2})+\vec{W_3})\geq -dim(\vec{V})$\\
& \\
& Similarly, $(\vec{W_1}+\vec{W_2})\subseteq \vec{V}$\\
& $\implies$ $dim(\vec{W_1}+\vec{W_2})\leq dim(\vec{V})$\\
& $\implies$ -$dim(\vec{W_1}+\vec{W_2})\geq -dim(\vec{V})$\\
& \\
\hline
& \\
& Considering these two inequations,\\
& -$dim((\vec{W_1} \cap \vec{W_2})+\vec{W_3})$-$dim(\vec{W_1}+\vec{W_2})$\\
& $\geq -2dim(\vec{V})$\\
& \\
& $\textbf{or,}$  $dim(\vec{W_1})+dim(\vec{W_2})+dim(\vec{W_3})$\\
& -$dim((\vec{W_1} \cap \vec{W_2})+\vec{W_3})$-$dim(\vec{W_1}+\vec{W_2})$\\
& $\geq dim(\vec{W_1})+dim(\vec{W_2})+dim(\vec{W_3})-2dim(\vec{V})$\\
& \\
& $\textbf{or,}$ $dim(\vec{W_1} \cap \vec{W_2} \cap \vec{W_3})$\\
& $\geq dim(\vec{W_1})+dim(\vec{W_2})+dim(\vec{W_3})-2dim(\vec{V})$\\
& \\
& $\implies$ $dim(\vec{W})\geq dim(\vec{W_1})+dim(\vec{W_2})$\\
& $+dim(\vec{W_3})-2dim(\vec{V})$\\
\hline
\multirow{3}{*}{} & \\
\textbf{Statement 3} & dim $\vec{W} \geq 7$ \\
\hline
Explanation & As $dim(\vec{W})\geq dim(\vec{W_1})+dim(\vec{W_2})$\\
& $+dim(\vec{W_3})-2dim(\vec{V})$\\
& $\implies dim(\vec{W})\geq$ (9+9+9) - (2$\times$10)\\
& $\implies dim(\vec{W}) \geq $ 7\\
\hline
$\textbf{Answer}$ & $7 \leq dim(\vec{W}) \leq 10$\\
& \\
\hline
\caption{$\textbf{Solution summary}$}
\label{table:1}
\end{longtable}
Hence, we can conclude that $dim(\vec{W}) \geq$ 7.
%\renewcommand{\theequation}{\theenumi}
%\begin{enumerate}[label=\thesection.\arabic*.,ref=\thesection.\theenumi]
%\numberwithin{equation}{enumi}
%\item Verification of the above problem using python code.\\
%\solution The  following Python code verifies the above solution.
%\begin{lstlisting}
%codes/multiplication_test.py
%\end{lstlisting}
%%%
%\end{enumerate}

\end{document}



